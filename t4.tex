\section{Partial Differential Equations. Finite difference method}

\begin{example}[(1D Heat equation)]\label{example4.0.1}
  \-\\Our problem is:
  \[
    \begin{cases}
      u_t - u_{xx} = f\\
      u(x,0) = u_0(x) &\longleftarrow\text{Initial condition (IC)}\\
      \begin{rcases}
        u(a,t) = u_a \\
        u(b,t) = u_b
      \end{rcases} &\longleftarrow\text{Boundary conditions (BC)}\\
    \end{cases}
  \]
  
  With $t\geq 0, \, x\in[a,b]$
\end{example}

We discretize $x$ and $t$:

\begin{figure}[h]
    \centering
    \begin{tikzpicture}[scale=0.7pt,line cap=round,line join=round,>=triangle  45,x=0.9555555555555555cm,y=1.011764705882353cm]
        \begin{axis}[
        x=0.9555555555555555cm,y=1.011764705882353cm,
        axis lines=middle,
        xmin=-1,
        xmax=8.5,
        ymin=-1,
        ymax=8.0,
        xtick={-0.0},
        ytick={-0.0},]
        \clip(-1.0,-1.0) rectangle (8.5,8.5);
        \draw (0.1,-0.01492590769987644) node[anchor=north west] {$a$};
        \draw (6.936006015235138,-0.04625816247894026) node[anchor=north west] {$b$};
        \draw (-1.18,3.765832835640491) node[anchor=north west] {$\Delta t$};
        \draw (3.1,-0.20291943637425935) node[anchor=north west] {$\Delta x$};
        \draw (0.8575485880967675,-0.01492590769987644) node[anchor=north west] {$x_1$};
        \draw (1.839292571174099,-0.03581407755258566) node[anchor=north west] {$x_2$};
        \draw (-0.7,1.2592524533153857) node[anchor=north west] {$t^1$};
        \draw (-0.7,2.2723286911717824) node[anchor=north west] {$t^2$};
        \draw (8.074411272207788,-0.12981084188977712) node[anchor=north west] {$\boldsymbol{x}$};
        \draw (-0.49,8) node[anchor=north west] {$\boldsymbol{t}$};
        \draw [->,line width=1.pt] (-0.3346034577059823,3.8604740672051325) -- (-0.3346034577059823,3.1204740672051328);
        \draw [->,line width=1.pt] (-0.3346034577059823,3.7017839952482134) -- (-0.3329059840463152,3.8906629755388313);
        \draw [->,line width=1.pt] (3.121531540966689,-0.14907205546295046) -- (3.948030184929092,-0.14559937208495718);
        \draw [->,line width=1.pt] (3.823001199491475,-0.1461247039565438) -- (3.0069329894929098,-0.15254473884094374);
        \draw [line width=0.5pt] (1.,0.) -- (1.,8.);
        \draw [line width=0.5pt] (2.,0.) -- (2.,8.);
        \draw [line width=0.5pt] (3.,0.) -- (3.,8.);
        \draw [line width=0.5pt] (4.,0.) -- (4.,8.);
        \draw [line width=0.5pt] (5.,0.) -- (5.,8.);
        \draw [line width=0.5pt] (6.,0.) -- (6.,8.);
        \draw [line width=0.5pt] (7.,0.) -- (7.,8.);
        \draw [line width=0.5pt,domain=0.0:8.5] plot(\x,{(--7.-0.*\x)/7.});
        \draw [line width=0.5pt,domain=0.0:8.5] plot(\x,{(--4.-0.*\x)/2.});
        \draw [line width=0.5pt,domain=0.0:8.5] plot(\x,{(--3.-0.*\x)/1.});
        \draw [line width=0.5pt,domain=0.0:8.5] plot(\x,{(--4.-0.*\x)/1.});
        \draw [line width=0.5pt,domain=0.0:8.5] plot(\x,{(--5.-0.*\x)/1.});
        \draw [line width=0.5pt,domain=0.0:8.5] plot(\x,{(--6.-0.*\x)/1.});
        \draw [line width=0.5pt,domain=0.0:8.5] plot(\x,{(--7.-0.*\x)/1.});
        \begin{scriptsize}
        \draw [fill=black] (1.,0.) circle (1.5pt);
        \draw [fill=black] (2.,0.) circle (1.5pt);
        \draw [fill=black] (3.,0.) circle (1.5pt);
        \draw [fill=black] (4.,0.) circle (1.5pt);
        \draw [fill=black] (5.,0.) circle (1.5pt);
        \draw [fill=black] (6.,0.) circle (1.5pt);
        \draw [fill=black] (7.,0.) circle (1.5pt);
        \draw [fill=black] (0.,0.) circle (1.5pt);
        \draw [fill=black] (0.,1.) circle (1.5pt);
        \draw [fill=black] (0.,2.) circle (1.5pt);
        \draw [fill=black] (0.,3.) circle (1.5pt);
        \draw [fill=black] (0.,4.) circle (1.5pt);
        \draw [fill=black] (0.,5.) circle (1.5pt);
        \draw [fill=black] (0.,6.) circle (1.5pt);
        \draw [fill=black] (0.,7.) circle (1.5pt);
        \draw [fill=black] (1.,1.) circle (1.5pt);
        \draw [fill=black] (1.,2.) circle (1.5pt);
        \draw [fill=black] (1.,3.) circle (1.5pt);
        \draw [fill=black] (1.,4.) circle (1.5pt);
        \draw [fill=black] (1.,5.) circle (1.5pt);
        \draw [fill=black] (1.,6.) circle (1.5pt);
        \draw [fill=black] (1.,7.) circle (1.5pt);
        \draw [fill=black] (2.,1.) circle (1.5pt);
        \draw [fill=black] (3.,1.) circle (1.5pt);
        \draw [fill=black] (4.,1.) circle (1.5pt);
        \draw [fill=black] (5.,1.) circle (1.5pt);
        \draw [fill=black] (6.,1.) circle (1.5pt);
        \draw [fill=black] (7.,1.) circle (1.5pt);
        \draw [fill=black] (2.,2.) circle (1.5pt);
        \draw [fill=black] (3.,2.) circle (1.5pt);
        \draw [fill=black] (4.,2.) circle (1.5pt);
        \draw [fill=black] (2.,3.) circle (1.5pt);
        \draw [fill=black] (2.,4.) circle (1.5pt);
        \draw [fill=black] (2.,5.) circle (1.5pt);
        \draw [fill=black] (2.,6.) circle (1.5pt);
        \draw [fill=black] (2.,7.) circle (1.5pt);
        \draw [fill=black] (3.,3.) circle (1.5pt);
        \draw [fill=black] (3.,4.) circle (1.5pt);
        \draw [fill=black] (3.,5.) circle (1.5pt);
        \draw [fill=black] (3.,6.) circle (1.5pt);
        \draw [fill=black] (3.,7.) circle (1.5pt);
        \draw [fill=black] (4.,3.) circle (1.5pt);
        \draw [fill=black] (4.,4.) circle (1.5pt);
        \draw [fill=black] (4.,5.) circle (1.5pt);
        \draw [fill=black] (4.,6.) circle (1.5pt);
        \draw [fill=black] (4.,7.) circle (1.5pt);
        \draw [fill=black] (5.,2.) circle (1.5pt);
        \draw [fill=black] (5.,3.) circle (1.5pt);
        \draw [fill=black] (5.,4.) circle (1.5pt);
        \draw [fill=black] (5.,5.) circle (1.5pt);
        \draw [fill=black] (5.,6.) circle (1.5pt);
        \draw [fill=black] (5.,7.) circle (1.5pt);
        \draw [fill=black] (6.,2.) circle (1.5pt);
        \draw [fill=black] (6.,3.) circle (1.5pt);
        \draw [fill=black] (6.,4.) circle (1.5pt);
        \draw [fill=black] (6.,5.) circle (1.5pt);
        \draw [fill=black] (6.,6.) circle (1.5pt);
        \draw [fill=black] (6.,7.) circle (1.5pt);
        \draw [fill=black] (7.,2.) circle (1.5pt);
        \draw [fill=black] (7.,3.) circle (1.5pt);
        \draw [fill=black] (7.,4.) circle (1.5pt);
        \draw [fill=black] (7.,5.) circle (1.5pt);
        \draw [fill=black] (7.,6.) circle (1.5pt);
        \draw [fill=black] (7.,7.) circle (1.5pt);
        \end{scriptsize}
        \end{axis}
        \end{tikzpicture}
    %\caption{}
    %\label{fig:my_label}
\end{figure}

\-\\\underline{Idea}: We'll impose $U_t(x_i,t^n) - U_{xx}(x_i,t^n) = f(x_i,t^n)$

\newpage

\subsection{Numerical derivatives}

\definecolor{qqwwzz}{rgb}{0.,0.4,0.6}

\begin{figure}[h]
    \centering
    \begin{tikzpicture}[line cap=round,line join=round,>=triangle 45,x=1.0cm,y=1.0cm]
    \begin{axis}[
    x=1.0cm,y=1.0cm,
    axis lines=middle,
    xmin=9.0,
    xmax=19.0,
    ymin=-1.0,
    ymax=7.0,
    xtick={0.0},
    ytick={0.0},y axis line style={draw = none}]
    \clip(9.,-1.) rectangle (19.,7.);
    \draw[line width=1.pt, smooth,samples=100,domain=0.0:0.11] plot[parametric] function{37.06*t**(3.0)+0.0*t**(2.0)+5.7*t+9.8,-38.54*t**(3.0)+0.0*t**(2.0)+7.51*t+3.02};
    \draw[line width=1.pt, smooth,samples=100,domain=0.11:0.15] plot[parametric] function{-52.47*t**(3.0)+28.91*t**(2.0)+2.59*t+9.91,39.61*t**(3.0)-25.24*t**(2.0)+10.22*t+2.92};
    \draw[line width=1.pt, smooth,samples=100,domain=0.15:0.24] plot[parametric] function{1.47*t**(3.0)+4.08*t**(2.0)+6.4*t+9.72,-18.89*t**(3.0)+1.69*t**(2.0)+6.09*t+3.13};
    \draw[line width=1.pt, smooth,samples=100,domain=0.24:0.3] plot[parametric] function{-13.11*t**(3.0)+14.44*t**(2.0)+3.95*t+9.91,9.63*t**(3.0)-18.57*t**(2.0)+10.89*t+2.76};
    \draw[line width=1.pt, smooth,samples=100,domain=0.3:0.35] plot[parametric] function{-7.62*t**(3.0)+9.58*t**(2.0)+5.38*t+9.77,-31.65*t**(3.0)+18.04*t**(2.0)+0.07*t+3.82};
    \draw[line width=1.pt, smooth,samples=100,domain=0.35:0.41] plot[parametric] function{-10.98*t**(3.0)+13.12*t**(2.0)+4.13*t+9.92,72.27*t**(3.0)-91.76*t**(2.0)+38.74*t-0.72};
    \draw[line width=1.pt, smooth,samples=100,domain=0.41:0.49] plot[parametric] function{3.59*t**(3.0)-4.8*t**(2.0)+11.48*t+8.91,47.73*t**(3.0)-61.58*t**(2.0)+26.37*t+0.97};
    \draw[line width=1.pt, smooth,samples=100,domain=0.49:0.57] plot[parametric] function{-20.86*t**(3.0)+30.92*t**(2.0)-5.91*t+11.73,40.31*t**(3.0)-50.75*t**(2.0)+21.09*t+1.83};
    \draw[line width=1.pt, smooth,samples=100,domain=0.57:0.62] plot[parametric] function{-10.43*t**(3.0)+13.18*t**(2.0)+4.14*t+9.84,-36.6*t**(3.0)+80.04*t**(2.0)-53.05*t+15.84};
    \draw[line width=1.pt, smooth,samples=100,domain=0.62:0.7] plot[parametric] function{28.35*t**(3.0)-59.13*t**(2.0)+49.1*t+0.52,-54.92*t**(3.0)+114.21*t**(2.0)-74.29*t+20.24};
    \draw[line width=1.pt, smooth,samples=100,domain=0.7:0.77] plot[parametric] function{-7.01*t**(3.0)+14.63*t**(2.0)-2.2*t+12.41,14.24*t**(3.0)-30.06*t**(2.0)+26.04*t-3.02};
    \draw[line width=1.pt, smooth,samples=100,domain=0.77:0.85] plot[parametric] function{36.51*t**(3.0)-85.69*t**(2.0)+74.89*t-7.34,-63.54*t**(3.0)+149.23*t**(2.0)-111.73*t+32.27};
    \draw[line width=1.pt, smooth,samples=100,domain=0.85:0.92] plot[parametric] function{-17.36*t**(3.0)+51.06*t**(2.0)-40.83*t+25.3,-8.72*t**(3.0)+10.06*t**(2.0)+6.03*t-0.95};
    \draw [line width=1.pt,dotted] (14.,0.)-- (13.99722993459645,4.732881120180119);
    \draw [line width=1.pt, >=latex, <->] (10.2,0.36)-- (13.6,0.36);
    \draw [line width=1.pt, >=latex, <->] (14.38,0.34)-- (17.88,0.34);
    \draw (11.76,1.06) node[anchor=north west] {$h$};
    \draw (15.9,1.04) node[anchor=north west] {$h$};
    \draw (9.52,0.) node[anchor=north west] {$x_{i-1}$};
    \draw (17.94,-0.04) node[anchor=north west] {$x_{i+1}$};
    \draw (13.84,0.) node[anchor=north west] {$x_{i}$};
    \draw (17.5,6.6) node[anchor=north west] {$f$};
    \draw (13,5.58) node[anchor=north west] {$f_i = f(x_i)$};
    \begin{scriptsize}
    \draw [color=qqwwzz] (14.,0.)-- ++(-1.5pt,0 pt) -- ++(3.0pt,0 pt) ++(-1.5pt,-1.5pt) -- ++(0 pt,3.0pt);
    \draw [fill=qqwwzz] (13.99722993459645,4.732881120180119) circle (1.5pt);
    \draw [color=black] (9.82,0.)-- ++(-1.5pt,0 pt) -- ++(3.0pt,0 pt) ++(-1.5pt,-1.5pt) -- ++(0 pt,3.0pt);
    \draw [color=black] (18.18,0.)-- ++(-1.5pt,0 pt) -- ++(3.0pt,0 pt) ++(-1.5pt,-1.5pt) -- ++(0 pt,3.0pt);
    \draw [fill=black,shift={(10.2,0.36)},rotate=90] (0,0) ++(0 pt,2.25pt) -- ++(1.9485571585149868pt,-3.375pt)--++(-3.8971143170299736pt,0 pt) -- ++(1.9485571585149868pt,3.375pt);
    \draw [fill=black,shift={(13.6,0.36)},rotate=270] (0,0) ++(0 pt,2.25pt) -- ++(1.9485571585149868pt,-3.375pt)--++(-3.8971143170299736pt,0 pt) -- ++(1.9485571585149868pt,3.375pt);
    \draw [fill=black,shift={(14.38,0.34)},rotate=90] (0,0) ++(0 pt,2.25pt) -- ++(1.9485571585149868pt,-3.375pt)--++(-3.8971143170299736pt,0 pt) -- ++(1.9485571585149868pt,3.375pt);
    \draw [fill=black,shift={(17.88,0.34)},rotate=270] (0,0) ++(0 pt,2.25pt) -- ++(1.9485571585149868pt,-3.375pt)--++(-3.8971143170299736pt,0 pt) -- ++(1.9485571585149868pt,3.375pt);
    \end{scriptsize}
    \end{axis}
    \end{tikzpicture}
    %\caption{}
    %\label{fig:my_label}
\end{figure}

We'll use Taylor's expansion on $f$ to find an approximation for the derivatives:
\begin{enumerate}[label={(\arabic*)}]
    \item $f_{i+1} = f_i + hf_i' + \frac{h^2}{2}f_i'' + \frac{h^3}{3!}f_i''' + \ldots$
    \item $f_{i-1} = f_i - hf_i' + \frac{h^2}{2}f_i'' - \frac{h^3}{3!}f_i''' + \ldots$
\end{enumerate}\-\\

\underline{Approximation of the first derivative}: \\
\begin{itemize}
    \item Forward differences (1) $$f_i' = \frac{f_{i+1} - f_i}{h} + \mathcal{O}(h)$$
    \item Backwards differences (2) $$f_i' = \frac{f_i-f_{i-1}}{h} + \mathcal{O}(h)$$
    \item Centered differences (1)-(2) $$f_i' = \frac{f_{i+1} - f_{i-1}}{2h} + \mathcal{O}(h^2)$$
\end{itemize}

\begin{remark}
  With centered differences, it converges faster, but sometimes the other two methods are preferable for ease in computations.
\end{remark}

\underline{Approximation of the second derivative} (1)+(2) $$f_i'' = \frac{f_{i+1} - 2f_i + f_{i-1}}{h^2} + \mathcal{O}(h^2)$$\\

There are higher order approximations, but these are the standard, and are more than enough. \\

Now, we'll present some notation to differentiate between the problem we want to solve and the numerical problem.

\begin{definition}\-\\
  The \textbf{continuous problem} is 
  $
    \begin{cases}
      \mathcal{L}(u) = f &\text{ in } \Omega, \, t > 0 \\
      \mathcal{B}(u) = q &\text{ in } \partial \Omega, t > 0 \\
      u(x,0) = u_0(x)
    \end{cases}
  $\\
  
  \-\\Where $\mathcal{L}$ is any differential operator. \\
  
  It's what we want to solve. \\
  
  The \textbf{discrete problem} \hspace{0.08cm}is\hspace{0.3cm}
  $
    \begin{cases}
      L(u) + \tau = f \\
      B(u) + \overline{\tau} = q \\
      u(x,0) = u_0(x)
    \end{cases}
  $\\
  
  \-\\The \textbf{numerical problem} is
  $
    \begin{cases}
      L(U) = f \\
      B(U) = q \\
      U(x,0) = u_0(x)
    \end{cases}
  $
\end{definition}

\-\\

Let's go back to our 1D heat equation: \\

The continuous problem is 

\[
\begin{cases}
  u_t - u_{xx} = f \quad t>0, x\in(a,b)\\
  u(a,t) = u_a\\
  u(b,t) = u_b \\
  u(x,0) = u_0(x)
\end{cases}
\]

\underline{Notation}: $u(x_i,t^n) = u_i^n$ \\
$x_i = a + i\Delta x \quad i = 0,\ldots N \quad x_0 = a,\, x_N = b \qquad$ (sometimes $\Delta x = h$).\\
$t^n = n\Delta t \, n\geq0$\\

\newpage

We have the following explicit method:

\subsection{Forward Time Centered Space method (FTCS)}

\begin{itemize}
    \item Approximate $u_t$ using forward differences (FT)
    \item Approximate $u_{xx}$ using centered differences (CD)
\end{itemize}

Our discretized problem is

\[
  \frac{u_i^{n+1} - u_i^n}{\Delta t} + \bigo(\Delta t) - \frac{u_{i+1}^n - 2u_i^n + u_{i-1}^n}{\Delta x^2} + \bigo(\Delta x^2) = f_i^n
\]

We define $\tau = \bigo(\Delta t,\Delta x^2)$ \\

Neglecting $\tau$, the numerical problem we want to solve is

\[
  \begin{cases}
    \dfrac{U_i^{n+1} - U_i^n}{\Delta t} - \dfrac{U_{i+1}^n - 2U_i^n + U_{i-1}^n}{\Delta x^2} = f_i^n \vspace{0.2cm}\\
    U_0^{n+1} = u_a(t^{n+1})\\
    U_N^{n+1} = u_b(t^{n+1}) \\
    U_i^0 = u_0(x_i)
  \end{cases}
\]

Now we define $r$ as $$r = \frac{\Delta t}{\Delta x^2}$$

So $$U_i^{n+1} = rU_{i-1}^n + (1-2r)U_i^n + rU_{i+1}^n + \Delta t f_i^n \qquad i = 1,\ldots,N-1$$

For $i=1$, $$U_1^{n+1} = (1-2r)U_1^n + rU_2^n + \Delta t f_1^n + ru_a^n$$

And for $i=N-1$, $$U_{N-1}^{n+1} = rU_{N-2}^n + (1-2r)U_{N-1}^n + \Delta t f_{N-1}^n + ru_b^n$$

Thus, $$\overline{U^{n+1}} = \underset{^\sim}{B}\overline{U^n} + \overline{F} + \overline{G}$$

Where 

$$\overline{U^p} = \left(U_1^p, U_2^p, \ldots , U_{N-1}^p \right)^T \qquad \overline{F}=\Delta t\left(f_1^n, f_2^n, \ldots,f_{N-1}^n \right)^T \qquad \overline{G} = \left(ru_a^n, 0, \ldots,0,ru_b^n \right)^T$$
$$
  \underset{^\sim}{B} = 
  \begin{pmatrix}
    \scriptstyle\small{1-2r} & r \\
    r    & \scriptstyle\small{1-2r} & r \\
         & \ddots & \ddots & \ddots \\
         &        & r & \scriptstyle\small{1-2r} & r \\
         &        &   & r    & \scriptstyle\small{1-2r}
  \end{pmatrix}
$$

\newpage

\begin{example}
  Suppose we have the following problem
  $$
  \begin{cases}
    u_t-\alpha u_{xx} = f \qquad x\in (a,b),\, t>0\\
    u(x,0) = u_0(x)\\
    \begin{rcases}
        u_x(a,t)=0\\
        u_x(b,t)=0
    \end{rcases}\,\text{Neumann boundary conditions}
  \end{cases}
  $$
  Now, $r:=\alpha\dfrac{\Delta t}{\Delta x^2}$ and our discretized problem is
  
  \[
    U_i^{n+1} = rU_{i-1}^n + (1-2r)U_i^n + rU_{i+1}^n \qquad i = 0,\ldots,N
  \]
  
  But for $i=0$ we have the term $rU_{-1}^n$ and $x_{-1}$ is located outside the domain! Could it be rewritten as something inside the domain?\\
  
  We can use the Neumann boundary conditions together with centered differences:
  
  $$
  \begin{rcases}
    \dfrac{du}{dx}\Big|_0 = g\\[10pt]
    \dfrac{du}{dx} \simeq \dfrac{u_1 - u_{-1}}{2h}
  \end{rcases}\implies u_{-1} = u_1 - 2hg
  $$
  
  in our case, $g=0$ so
  
  \[
    U_0^{n+1} = (1-2r)U_0^n + 2rU_1^n + \cancelto{0}{(-2rhg)}
  \]
  
\end{example}

\begin{remark}
  We have the same situation for $i = N$ and it's solved similarly.
\end{remark}


\subsection{$\theta$-methods}

Now we'll proceed to deduce the $\theta$-methods, which are obtained by approximating the time derivative at $t^{n+\theta}$. We will work with the problem in example \ref{example4.0.1}.\\

We expand around $t+\theta\Delta t$
\begin{align*}
    &f^n = f(t^n) = f(t^n+\theta\Delta t) - \theta\Delta tf'(t^n+\theta \Delta t) + (\theta\Delta t)^2 f''(t^n + \theta\Delta t) + \mathcal{O}(\Delta t^3)\\
    &f^{n+1} = f(t^n+\Delta t) = f(t^n + \theta\Delta t) + (1-\theta)\Delta t f'(t^n + \theta\Delta t) + (1-\theta)^2\Delta t^2 f''(t^n + \theta\Delta t) + \mathcal{O}(\Delta t^3)
\end{align*}
and subtracting, we obtain
\begin{align*}
    f^{n+1} - f^n = \Delta tf'(t^n + \theta\Delta t) + (\Delta t^2 - 2\theta\Delta t^2)f''(t^n + \theta\Delta t) + \mathcal{O}(\Delta t^3)\\
    \frac{f^{n+1} - f^n}{\Delta t} = f'(t^n + \theta\Delta t) + \Delta t (1 - 2\theta)f''(t^n + \theta\Delta t) + \mathcal{O}(\Delta t^2)
\end{align*}

\begin{remark}
  This is second order if $\theta = \frac{1}{2}$. First order otherwise.
\end{remark}

So

\[
  \frac{\partial f}{\partial t}\Bigg|_i^{n+\theta} = \frac{f^{n+1} - f^n}{\Delta t} + \tau
\]

where $\tau = \begin{cases}
  \mathcal{O}(\Delta t^2) &\,\text{ if } \theta = \frac{1}{2}\\
  \mathcal{O}(\Delta t) & \text{otherwise}
\end{cases}$\\

Similarly, adding (and using the previous result), we obtain

\[
   f^{n+\theta} = \theta f^{n+1} + (1-\theta)f^n + \mathcal{O}(\Delta t^2)
\]

We can't use function values that are not in positions $n$ or $n+1$, which are the only one we know. That's why we rewrite the values at $n+\theta$ as values at $n$ and $n+1$.\\

With that, we end up with

\[
  u_t(x_i,t^{n+\theta}) = \frac{u(x_i,t^{n+1}) - u(x_i,t^n)}{\Delta t} + \tau
\]

\begin{align*}
  u_{xx}(x_i,t^{n+\theta}) &= \theta u_{xx}(x_i,t^{n+1}) + (1-\theta)u_{xx}(x_i,t^n) + \mathcal{O}(\Delta t^2) \underset{\substack{\text{approx. 2nd}\\\text{order deriv.}}}{=}\\
  &= \theta \frac{u_{i+1}^{n+1} - 2u_i^{n+1} + u_{i-1}^{n+1}}{\Delta x^2} + (1-\theta)\frac{u_{i-1}^n - 2u_i^n + u_{i+1}^n}{\Delta x^2} + \mathcal{O}(\Delta x^2) + \mathcal{O}(\Delta t^2)
\end{align*}

So, the numerical problem we have to solve is (in matrix form)

\[
  \subTilde{A}\Bar{U}^{n+1} = \subTilde{B}\Bar{U}^n + \Bar{F} + \Bar{G}
\]

where

\[
  \Bar{U}^n = \left( U_1^n, U_2^n, \ldots, U_{N-1}^n \right)
\]

\[
  \subTilde{A} = \begin{pmatrix}
    \scriptstyle{\small 1+2r\theta} & -r\theta              \\
    -r\theta   & \scriptstyle{\small 1+2r\theta} & -r\theta \\
    &\ddots     & \ddots     & \ddots
  \end{pmatrix} \qquad \subTilde{B} = \begin{pmatrix}
    \scriptstyle{\small 1-2r(1-\theta)} & \scriptstyle{\small r(1-\theta)}            \\
    \scriptstyle{\small r(1-\theta)}   & \scriptstyle{\small 1-2r(1-\theta)} & \scriptstyle{\small r(1-\theta)} \\
    \hspace{1.7cm}\ddots  & \hspace{1.9cm} \ddots     & \hspace{1.7cm} \ddots
  \end{pmatrix}
\]

\[
  \Bar{G} = r\begin{pmatrix}
    \theta u_a^{n+1} + (1-\theta)u_a^n\\[3pt]
    0\\[3pt]
    \vdots\\[3pt]
    0\\[3pt]
    \theta u_b^{n+1} + (1-\theta)u_b^n
  \end{pmatrix} \qquad \Bar{F}=\Delta t\begin{pmatrix}
    \theta f_1^{n+1} + (1-\theta)f_1^n\\[3pt]
    \theta f_2^{n+1} + (1-\theta)f_2^n\\[3pt]
    \theta f_3^{n+1} + (1-\theta)f_3^n\\[3pt]
    \theta f_4^{n+1} + (1-\theta)f_4^n\\[3pt]
    \vdots
  \end{pmatrix}
\]
\-\\
\begin{remark}\-
  \begin{itemize}
      \item If $\theta = 0$ we recover the \textbf{explicit method} (FTCS).
      \item If $\theta = 1$ we have the \textbf{implicit method} (Backward Time Centered Space - BTCS).
      \item If $\theta = \frac{1}{2}$ we obtain the \textbf{Crank-Nicolson method}.
  \end{itemize}
  \-\\
  Implicit and Crank-Nicolson methods are always stable regardless of $\Delta t$, but a system of equations has to be solved at each step.
\end{remark}

\begin{definition}
  
\end{definition}