\section{Stiff Problems}

In some ODEs, the step size taken by an adaptive method is forced to be unreasonably small even in regions where the solution curve is smooth. In these cases, it takes a large amount of steps to go through a short time interval.\\

These types of equations are called \textbf{stiff ODEs}.

\begin{example}[(Van der Pol equation)]\-\\
Given the Van der Pol equation $$\ddot{x} - \mu(1-x^2)\dot{x} + x = 0 \qquad (\text{with $x(0) = 2$})$$ 

the larger the constant $\mu$, the stiffer is the problem. \\

Trying to solve it using an explicit adaptive stepsize method like Matlab's \texttt{ode45} yields


\begin{figure}[h]
  \centering
  \includesvg{VanDerPolOde45.svg}
  \caption{Van der Pol equation solution with ode45 ($\mu = 10$)}
\end{figure}

With $873$ steps needed, and a minimum stepsize of $2.5119\cdot 10^{-5}$.\\

\newpage

Now, using an implicit method like Matlab's \texttt{ode15s}, we have

\begin{figure}[h]
  \centering
  \includesvg{VanDerPolOde15s.svg}
  \caption{Van der Pol equation solution with ode15s ($\mu = 10$)}
\end{figure}

Which clearly uses less steps to pass through the stiff areas (a total of $326$ with minimum stepsize $0.00014607$).\\

If we where to solve it with a larger $\mu$, for example $\mu = 1000$, the number of steps needed using \texttt{ode45} is $5.495.393$ which is too much compared to the $586$ needed with \texttt{ode15s}.\\

\end{example}

\newpage

\begin{definition}
  The set of values of the stepsize $h$ such that $\lim\limits_{h\to0}y_n = 0$ is called the \textbf{absolute stability region} ($\Omega$).
\end{definition}

Let's see what happens with a general linear multistep method applied to a linear system:\\

Our method is

\[
  \underbrace{\sum_{j=0}^k\alpha_jy_{n+j}}_{\rho} = h\underbrace{\sum_{j=0}^k\beta_jf_{n+j}}_{\sigma}
\]

We apply it to $$y' = Ay$$ where we'll assume that the eigenvalues of $A$ ($\lambda_1, \ldots, \lambda_n$) are all different (so that it's diagonalizable) and have negative real parts. \\

$y = e^{Ax}$ is the fundamental solution, and if $y(0) = y_0$, the solution is $$y(x) = e^{Ax}y_0$$ with $$\lim_{x\to\infty}e^{Ax} = \begin{pmatrix}
                                               0      & \ldots & 0      \\
                                               \vdots & \ddots & \vdots \\
                                               0      & \ldots & 0
                                             \end{pmatrix}$$

Now, with the system $y' = Ay$, our general method looks like

\[
  \sum_{j=0}^k\alpha_jy_{n+j} = h\sum_{j=0}^k\beta_jAy_{n+j}
\]

Let's rewrite it:

$$\sum_{j=0}^k(\alpha_jI-h\beta_jA)y_{n+j} = 0$$
\begin{center}
    \rotatebox{270}{$\leadsto$}
\end{center}
$$\sum_{j=0}^k\left(\alpha_jI-h\beta_j\begin{pmatrix}\lambda_1 & & \\ & \ddots & \\ & & \lambda_n \end{pmatrix}\right)y_{n+j} = 0$$
\begin{center}
    \rotatebox{270}{$\leadsto$}
\end{center}
$$\sum_{j=0}^k(\alpha_jI-h\beta_j\lambda_i)y_{n+j} = 0 \qquad (\forall i = 1,\ldots,n)$$

\newpage

A finite difference equation looks like

\[
  a_ky_{n+k} + a_{k-1}y_{n+k-1} + \ldots + a_0y_n = 0
\]

Which has roots $r_1,\ldots,r_k$. So the solution is:

\[
  y_n = c_1r_1^n + c_2r_2^n + \ldots + c_kr_k^n \underset{\text{if } Re(r_k)<0}{\xrightarrow{\hspace*{2cm}}} 0
\]

\begin{prop}
    Given $$\hat{h} = h\lambda_i, \quad \rho(z) = \sum\alpha_jz^j, \quad \sigma(z) = \sum\beta_jz^j$$
    the method is absolutely stable for values of $\hat{h}$ such that the roots of $\rho(z) - \hat{h}\sigma(z) = 0$ have negative real parts.
\end{prop}
\-\\
Let's see the one dimensional case with Euler's method:

\usetikzlibrary[patterns]

\begin{example}[(Stability region with Euler's method)]
    
    \[
      \begin{rcases}
        y_{n+1} = y_n + hf(y_n)\\
        \hspace{2cm}f(y) = \lambda y
      \end{rcases} \leadsto y_{n+1} = y_n + h\lambda y_n = (1+h\lambda)y_n
    \]
    
    $$\implies y_n = y_0c(1+h\lambda)^n \underset{\text{if } \abs{1+\lambda h} < 1}{\xrightarrow{\hspace*{2cm}}} 0$$
    \begin{figure}[h]
        \definecolor{qqwwzz}{rgb}{0.,0.4,0.6}
        \centering
        \begin{tikzpicture}[line cap=round,line join=round,>=triangle 45,x=1.0cm,y=1.0cm, scale = 2]
        \begin{axis}[
        x=1.0cm,y=1.0cm,
        axis lines=middle,
        xmin=-3.0,
        xmax=1.0,
        ymin=-1.5,
        ymax=1.5,
        xtick={-0},
        ytick={-0},]
        \clip(-3.,-1.5) rectangle (1.,1.5);
        \draw [rotate around={0.:(-1.,0.)},line width=0.5pt,color=qqwwzz,fill=qqwwzz,pattern=north east lines,pattern color=qqwwzz] (-1.,0.) ellipse (1.cm and 1.cm);
        \draw (-1,0.8) node[anchor=north west] {$\Omega$};
        \draw (-1.4,0.012479994921094003) node[anchor=north west] {$\scriptstyle\footnotesize{-1}$};
        \begin{scriptsize}
        \draw [color=black] (-1.,0.)-- ++(-1.5pt,0 pt) -- ++(3.0pt,0 pt) ++(-1.5pt,-1.5pt) -- ++(0 pt,3.0pt);
        \end{scriptsize}
        \end{axis}
        \end{tikzpicture}
        \caption{For all eigenvalues, $\hat{h}$ must be in $\Omega$ (stability region)}
        \label{fig:my_label}
    \end{figure}
    
    \newpage
    
    \underline{Particular example}\\
    
    $$\begin{cases}
        y_1' = -2y_1 + y_2 + 2\sin t \\
        y_2' = 998y_1 - 999y_2 + 999(\cos t - \sin t)
      \end{cases}$$
    
    This is a stiff system, with
    
    \[
      J = 
      \begin{pmatrix} 
        -2  & 1 \\
        998 & -999
      \end{pmatrix} \qquad \text{Eigenvalues}\left(J \right) = \setb{-1000, -1}
    \]\\
    So the stepsize $h$ needs to verify $$\abs{1+h(-1000)} < 1$$
    
    $$\implies h \leq \frac{1}{500}$$
\end{example}
\-\\
Let's do it now with backwards Euler:

\begin{example}[(Stability region with backwards Euler)]
    $$y_{n+1} = y_n + \hat{h}y_{n+1} \leadsto (1-\hat{h})y_{n+1} = y_n \leadsto y_{n+1} = \frac{1}{1-\hat{h}}y_n$$
    $$\implies \norm{1-\hat{h}} = \norm{\hat{h}-1}>1$$
    
    \begin{figure}[h]
        \centering
        \definecolor{qqwwzz}{rgb}{0.,0.4,0.6}
        \begin{tikzpicture}[line cap=round,line join=round,>=triangle 45,x=1.0cm,y=1.0cm, scale = 2]
        \begin{axis}[
        x=1.0cm,y=1.0cm,
        axis lines=middle,
        xmin=-1.0,
        xmax=3.0,
        ymin=-1.5,
        ymax=1.5,
        xtick={-0},
        ytick={-0},]
        \clip(-1.,-1.5) rectangle (3.,1.5);
        \fill[pattern=north east lines,pattern color=qqwwzz]
        (current bounding box.north west)
        rectangle
        (current bounding box.south east);
        \draw [rotate around={0.:(1.,0.)},line width=0.5pt,color=qqwwzz,fill=white] (1.,0.) ellipse (1.cm and 1.cm);
        \draw (0.79,0.03359130568751314) node[anchor=north west] {$\scriptstyle\footnotesize{1}$};
        \draw (2.0233108635400727,1.3) node[anchor=north west] {$\Omega$};
        \begin{scriptsize}
        \draw [color=black] (1.,0.)-- ++(-1.5pt,0 pt) -- ++(3.0pt,0 pt) ++(-1.5pt,-1.5pt) -- ++(0 pt,3.0pt);
        \draw (0,0) -- (2,0);
        \end{scriptsize}
        \end{axis}
        \end{tikzpicture}
    \caption{Stability region $\Omega$ with backwards Euler}
    \label{fig:my_label}
    \end{figure}
    
\end{example}

\begin{remark}
      If all eigenvalues are $< 1$, we can take any $h$, as all $\hat{h}$ are $<0$.
\end{remark}

\newpage

\begin{example}[(Dekker-Verwer p.7-8)]
    
\end{example}